%!TEX program = xelatex
%使用xelatex,使用unicode,对中文支持的最好.记得文档另存为uft-8格式.

\documentclass[10pt,a4paper,titlepage]{article} %默认字体大小,页面大小(a4),标题单独一页
\usepackage{fontspec}   %引入设置字体的包
\usepackage{xeCJK}  %使用中文环境的包,还有一个是CJK,未知
\usepackage{indentfirst}
\setlength{\parindent}{2em}   %首行缩进两字符
\renewcommand{\contentsname}{\centerline{目录}}


\setmainfont{Times New Roman}   %英文字体
\setCJKmainfont[BoldFont=MicrosoftYaHei]{MicrosoftYaHei}      %中文字体`'
\title{XXX产品软件概要设计书}
\author{}
\date{}

\begin{document}
\maketitle

\tableofcontents  %生成目录
\clearpage
\section{引言}
\subsection{背景}
[本文档的简要功能说明。]\newline
[本文档适用于哪些人员、哪些项目、哪些领域等。]\newline
[简要说明该产品的市场背景和主要特点。]\newline
\subsection{产品信息}
产品名称:\newline
产品型号:\newline
\subsection{软件名称}
[说明本文档对应的软件的正式名称和版本号,或各部分的文件名和版本号。]\newline
\subsection{术语和缩略语}
[对文中使用的术语和缩略语进行说明。]\newline

\subsection{参考资料}

\section{总体设计}
\subsection{需求规定}
[说明对本系统的主要的输入输出项目、处理的功能性能要求。]\newline
\subsection{运行环境}
[简要地说明对本系统的运行环境(包括硬件环境和支持环境)的规定]\newline
\subsection{开发环境}
[详细说明开发和调试/测试该软件所需的硬件环境和软件环境。]\newline
\subsection{设计思想}
[针对需求,说明软件的全局设计思想(如面向实时、面向对象、面向数据、事件驱动等)和实现方法。]\newline
\subsection{系统结构}
[用一览表及框图的形式说明本系统的系统元素(各层模块、子程序、公用程序等)的划分,扼要说明每个系统元素的标识符和功能,分层次地给出各元素之间的关系(控制、顺序、信号传递等关系。)]\newline
\subsection{处理流程}
[说明本系统的处理流程,包括信息处理流程、功能实现过程、资源调配过程和进程控制流程等。硬使用图表的形式说明。] \newline
\subsection{功能实现与模块的关系}
[用如下矩阵图说明各项功能需求的实现同各模块的分配关系:\newline
\subsection{模块开发方式说明}
[用列表说明各模块的开发方式(新开发、移植、改进、直接使用、外协开发等)。]\newline
\section{接口设计}
\subsection{外部接口}
\subsubsection{用户接口}
[说明软件的用户界面设计要求。]
\subsubsection{硬件接口}
[陈述软件产品与系统硬件设备之间每一个接口的逻辑特点。还可能包括如下事宜: 要支持什么样的设备,如何支持这些设备,如何约定等。]
\end{document}